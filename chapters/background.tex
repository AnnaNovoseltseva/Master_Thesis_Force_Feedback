\chapter{Background}
\label{back}

\section{Teleoperated Surgical Robots}
\label{sec:daVinci}
Recently robots started to be extensively used for surgical procedures. The use of robots allows doctors to perform surgical procedures with high accuracy, repeatability and reliability. Which in turn results in reducing operation time, errors and post-operation injuries. Minimally invasive surgeries are beneficial for accurate procedures with minimal access to operated organs, e.g. neurosurgery, eye surgery, cardiac surgery, intravascular surgeries and etc.  Use of robots in minimally invasive procedures improves precision and reliabilty of operations. \cite{tavakoli_haptics_2008}

There are two types of devices used for surgeries, supporting and augmenting. 
Supporting devices perform secondary functions to support the surgeon.  Some of them used for positioning and stabilization purposes of cameras, endoscopic tools, ultrasound probes and etc. Others to increase device dexterity or autonomy (dexterous and autonomous endoscopes).

Augmenting devices are used to extend surgeon's ability in performing an operation. They can be divided in four categories. Hand-held tools are augmenting instruments that used for hand tremors reduction, for dexterity and navigation capability increase. Another type of augmenting devices are cooperatively-controlled tools, where the surgeon and the robot cooperatively manipulate the surgical device (e.g. ROBODOC system, Steady- Hand robot, LARS, the Neurobot, and the ACROBOT system). Teleoperated robots are type of augmenting tools, where surgeon (master) controls the movements of a surgical robot (slave) via a surgeon's console (e.g. the daVinci and the Zeus systems). And autonomous tools, which can perform some tasks (suturing and knot tying) autonomously. \cite{tavakoli_haptics_2008}

Use of teleoperated robots in surgeries can solve many of the conventional surgery problems in terms of more precise manipulation capability, ergonomics, dexterity, and haptic feedback capability for the surgeon. They enhance dexterity by increase of instrument degrees of freedom, hand tremor compensation, and movements scaling that allows transformation of the control grips large movements into small motions inside the patient. 3-D view with depth perception gives surgeons ability to directly control a stable visual field with increased magnification and maneuverability. All of these enhances the surgeon's operation performance. However, robot-assisted surgeries are high cost, need large operational room space, do not have established efficacy, and need for tableside assistants. For these reasons ability of hospitals to use surgical robots is low, making their use for routine surgeries improbable. \cite{tavakoli_haptics_2008}

Today, many surgical robotic systems have been commercially developed and approved by the FDA, such as the daVinci surgical system (Intuitive Surgical, Inc., Sunnyvale, CA) , Sensei X robotic catheter system (Hansen Medical Inc., Mountain View, CA), FreeHand v1.2 (FreeHand 2010 Ltd., Cardiff, UK), Invendoscopy E200 system (Invendo Medical GmbH, Germany), Flex® robotic system (Medrobotics Corp., Raynham, MA), Senhance (TransEnterix, Morrisville, NC), Auris robotic endoscopy system (ARES; Auris Surgical Robotics, Silicon Valley, CA, USA), The NeoGuide Endoscopy System (NeoGuide Endoscopy System Inc, Los Gatos, CA). \cite{lanfranco_robotic_2004,peters_review_2018} 

There is also number of NON-FDA-approved platforms that currently under development or going through clinical trials. For example, MiroSurge (RMC, DLR, German Aerospace Center, Oberpfaffenhofen-Weßling), The ViaCath system (BIOTRONIK, Berlin, Germany), SPORT™ surgical system (Titan Medical Inc., Toronto, Ontario), The SurgiBot™ (TransEnterix, Morrisville, NC), The Versius Robotic System (Cambridge Medical Robotics Ltd., Cambridge, UK), MASTER (Nanyang Technological University and National University Health System), Verb Surgical (Verb Surgical Inc., J \& J/Alphabet, Mountain View, CA, USA), Miniature in vivo robot (MIVR) (MIVR, Virtual Incision, CAST, University of Nebraska Medical Center, Omaha, Nebraska, USA), the Einstein surgical robot (Medtronic, Minneapolis, MN). \cite{peters_review_2018}

The daVinci® Surgical System is one of the most commonly used robotic surgical systems. In 2015, over 3400 systems were in use around the world. More than 3 million surgeries were performed worldwide using daVinci system \cite{_intuitive_2018}. The system has been approved for various types of surgeries such as cardiac, colorectal, thoracic, urological and gynecologic. However, new systems are emerging on the market, providing features that are absent currently in the daVinci System. For example, in 2017 FDA approved Senhance robotic platform that provides actual haptic force feedback, allowing the surgeon to feel forces generated at the instruments end. In addition, the system uses eye-tracking technology to move the camera at the point the surgeon is looking at, while the daVinci uses a footswitch panel to control the camera movement. Another example is Flex Robotic System, which consists of flexible endoscope for laparoendoscopic surgeries. This system is able to define a non-linear path to surgical target by advancing a flexible telescopic inner-outer mechanism with instruments inside it, whereas instruments in the daVinci system can follow only non-flexible straight path. \cite{peters_review_2018}


Effectiveness of daVinci system in comparison to opened surgery and other systems \cite{yu_safety_2014}



\section{Importance of Haptic Feedback}
\label{sec:hapticFeedbackImportance}
Write about studies with and without haptic feedback.

It has been shown that incorporating force feedback into teleoperated sys- tems can reduce the magnitude of contact forces and therefore the energy consumption, the task completion time and the number of errors. In sev- eral studies [122, 147, 15], addition of force feedback is reported to achieve some or all of the following: reduction of the RMS force by 30% to 60%, the peak force by a factor of 2 to 6, the task completion time by 30% and the error rate by 60%. 

In [106], a scenario is proposed to incorporate force feedback into the Zeus surgical system by integrating a PHANToM haptic input device into the system. In [85], a dextrous slave combined with a modified PHAN- ToM haptic master which is capable of haptic feedback in four DOFs is presented. A slave system which uses a modified Impulse Engine as the haptic master device is described in [30]. In [107], a telesurgery master- slave system that is capable of reflecting forces in three degrees of freedom (DOFs) is discussed. A master-slave system composed of a 6-DOF parallel slave micromanipulator and a 6-DOF parallel haptic master manipulator is described in [150]. Other examples of haptic surgical teleoperation include [93] and [11]. The haptics technology can also be used for surgical training and simulation purposes. For example, a 7-DOF haptic device that can be applied to surgical training is developed in [56]. A 5-DOF haptic mecha- nism that is used as part of a training simulator for urological operations is discussed in [146]. 


\section{Current Approaches}
\label{sec:CurAppr}

1.3 Haptics for Robotic Surgery and Therapy 
Incorporating haptic sensation to robotic systems for surgery or therapy especially for minimally invasive surgery, which involves limited instrument maneuverability and 2-D camera vision, is a logical next step in the develop- ment of these systems. To do so, in addition to instrumentation of surgical tools, appropriate haptics-enabled user interfaces must be developed.
 
1.3.1 Haptic user interface technology
In the following, examples of the currently available haptic devices are 
described. For a more complete survey of haptic devices, see [55]. 

1.3.1.1 PHANToM 
The PHANToM from Sensable Technologies Inc. (www.sensable.com) is one of the most commonly used haptic devices and comes in a number of models with different features. PHANToM 1.5A provides six DOFs input control. Of the six DOFs of the arm, depending on the model, some or all are force-reflective. In Figure 1.2a, a PHANToM 1.5/6DOF with force feedback capability in all of the six DOFs is shown. 

1.3.1.2 Freedom-6S 
The Freedom-6S shown in Figure 1.2b is a 6-DOF device from MPB Technologies Inc. (www.mpb-technologies.ca) that provides force feedback in all of the six degrees of freedom. The position stage is direct driven while the orientation stage is driven remotely by tendons. The Freedom-6S features static and dynamic balancing in all axes (see [54] for further design details). 
1.3.1.3 Laparoscopic Impulse Engine and Surgical Workstation 
Originally as part of a laparoscopic surgical simulator, the Laparoscopic Impulse Engine was designed by Immersion Corp. (www.immersion.com). The device can track the position of the instrument tip in five DOFs with high resolution and speed while providing force feedback in three DOFs. More recently, Immersion has developed the Laparoscopic Surgical Work- station (Figure 1.2c), which is capable of providing force feedback in five DOFs. An application example is the Virtual Endoscopic Surgery Trainer (VEST) from Select-IT VEST Systems AG (www.select-it.de). The VEST system uses the Laparoscopic Impulse Engine as its force-feedback input interface for simulating laparoscopic surgery interventions. 

1.3.1.4 Xitact IHP 
The Xitact IHPTM from Xitact Medical Simulation (www.xitact.com) is a 4-DOF force feedback manipulator based on a spherical remote-center-of- motion mechanical structure and was originally designed for virtual reality based minimally invasive surgery simulation [42]. It features high output force capability, low friction, zero backlash and a large, singularity-free workspace. A picture of the Xitact IHP is shown in Figure 1.2d. 



Placing force sensors on the surgical instrument \cite{hong_design_2012}

They suggest to the measure of pulling and grasp forces at the tip of surgical 
instrument. For the design of the compliant forceps, the required compliance 
characteristics are first defined using a simple spring model with one linear 
and one torsional springs. This model may be directly realized as the compliant 
forceps. However, for the compact realization of the mechanism, we synthesize 
the spring model with two torsional springs that has equivalent compliance 
characteristics to the linear-torsional spring model. Then, each of the 
synthesized torsional springs is realized physically by means of a flexure 
hinge. From this design approach, direct measurement of the pulling and grasp 
forces is possible at the forceps, and measuring sensitivity can be adjusted 
in the synthesis process. The validity of the design is evaluated by finite 
element analysis. Further, from the measured values of bending strains of two 
flexure hinges, a method to compute the decoupled pulling and grasp forces is 
presented via the theory of screws. Finally, force- sensing performance of the 
proposed compliant forceps is verified from the experiments of the prototype 
using some weights and load cells. 10.1109/TRO.2012.2194889 

Making new surgical instrument design \cite{schwalb_forcesensing_2017}

Method In this paper a force-feedback enabled surgical robotic system is 
described in which the developed force-sensing surgical tool is discussed in 
detail. The developed surgical tool makes use of a proximally located 
force/torque sensor, which, in contrast to a distally located sensor, 
requires no miniaturization or sterilizability. Results Experimental results 
are presented, and indicate high force sensing accuracies with errors <0.09 N. 
Conclusions It is shown that developing a force-sensing surgical tool utilizing 
a proximally located force/torque sensor is feasible, where a tool outer diameter 
of 12 mm can be achieved. For future work it is desired to decrease the current 
tool outer diameter to 10 mm. 

Sensorless estimation methods

Vision based solution \cite{aviles_towards_2017}

They proposed to use vision based solution with supervised learning to estimate 
the applied force and provide the surgeon with a suitable representation of it.
 The proposed solution starts with extracting the geometry of motion of the 
 heart's surface by minimizing an energy functional to recover its 3D deformable 
 structure. A deep network, based on a LSTM-RNN architecture, is then used to 
 learn the relationship between the extracted visual-geometric information and 
 the applied force, and to find accurate mapping between the two. Our proposed 
 force estimation solution avoids the drawbacks usually associated with force 
 sensing devices, such as biocompatibility and integration issues. We evaluate 
 our approach on phantom and realistic tissues in which we report an average 
 root-mean square error of 0.02 N.

SLiding pertrubation observer

This paper suggests a bilateral controller applying sliding perturbation 
observer based force estimation method. In the suggested bilateral controller, 
the master control uses impedance control and the slave control uses a sliding 
mode control (SMC). A torque and force sensorless teleoperation system can be 
implemented using the suggested bilateral control structure through an 
experimental evaluation. This paper presents a method of estimating the 
reaction force of the surgical robot instrument without sensors and attempts 
to use state observer of control algorithm. Sliding mode control with sliding 
perturbation observer (SMCSPO) is used to drive the instrument, where the 
sliding perturbation observer (SPO) computes the amount of perturbation 
defined as the combination of the uncertainties and nonlinear terms where 
the major uncertainties arise from the reaction force. Based on this idea, 
this paper proposes a method to estimate the reaction force on the end-effector 
tip of the surgical robot instruments using only SPO and encoder without 
any additional sensors. To evaluate the validity of this paper, experiment 
was performed and the results showed that the estimated force computed from 
SPO is similar to the actual force.

Measuring the proximal guide wire force \cite{yoon_design_2015}

They measure the proximal guide wire force and the force between the surgeon's 
hand and the handle used on the Phantom, the force feedback closed-loop control 
can effectively eliminate the loss of mechanical impedance of force feedback 
information. The accuracy control of force feedback is greatly enhanced in the 
aspect of security and the operation efficiency.

Write about all disadvantages of previous methods.

Tools - >limited lifetime - find citation

Vision -> huge time delays, accuracy

Wire force - > repeatability issue


\section{Force Sensors}
\label{sec:ForceSensors}

Two general principles dominate in force measurement: piezoelectric and strain-gauge sensors. \cite{SGandP1}

Piezoelectric sensors consist of two crystal disks with an electrode foil in between. When force is applied, an electric charge, proportional to the applied force, is obtained and can be measured. Piezoelectric sensors show small deformation when force is applied, this results in a high resonance frequency. Also, piezoelectric sensors due to their principle of operation have significant linearity error and drift. \cite{SGandP2}

In the strain gauge based force transducers the force causes deformation and subsequent linear change in resistance. Strain gauges are usually connected to a Wheatstone bridge circuit, where the output voltage is proportional to the applied force. Strain gauge based transducers provide small individual errors (200 ppm), show no drift, and are therefore appropriate for long-term monitoring tasks. However, they are relatively big, temperature dependent, and have lower resonance frequency in comparison to piezoelectric sensors. \cite{SGandP1,SGandP2}

On the basis of the above mentioned, piezoelectric sensors are preferable for dynamic measurements of small forces while strain gauge sensors are better when large forces are measured. In this study, strain gauges were used since they show better accuracy and long-term stability. \cite{SGandP1,SGandP2}

We can use QTC-pills, but it has no-linearity and hysteresis issues. We can test in the future.


\section{Contributions}
\label{sec:MyAppr}
Force sensing devices for measuring forces in X-Y direction and one for Z-direction measurement were created. They allow to get accurate force readings from the daVinci tools of the PSM.  These devices can be easily added to the existing daVinci system. Since we have to add created device on each robot arm only, it is cheaper than placement of sensors on each separate surgical tool.  Moreover, created devices allow to get force data faster than through visual data processing method. 
