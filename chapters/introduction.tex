\chapter{Introduction}
\label{intro} % Always give a unique label


Common ground. Teleoperated da Vinci surgical system is a robot-assisted surgical system that enhances surgeons performance in minimally invasive surgeries by allowing highly precise translation of surgeon`s hand movements to the instrument's movements. 

Practical problem condition. The currently available da Vinci surgery system has a laparoscopic camera, providing visual feedback to guide doctors during surgery. However, the system does not have any kinesthetic or cutaneous feedback, known as haptics.\cite{_intuitive_2018} 

The costs of that condition. 
During open surgeries, doctors usually get haptic feedback directly or through the surgical tools. In minimally invasive surgeries interaction with patients via long shafts leads to the loss of some force and tactile sense. In robotic surgery systems, surgeons have to manipulate robots indirectly, which leads to an elimination of any haptic feedback. \cite{okamura_haptic_2009} 

It is believed that the addition of haptic feedback in the da Vinci surgery robot will help to reduce the amount of surgical errors and intra-operative injuries, which will lead to faster post-surgery recovery time and decreased rate of unsuccessful surgeries. \cite{reiley_effects_2008, van_der_meijden_value_2009, okamura_haptic_2009}

Gist of your solution. There are many technical challenges to overcome in order to implement the haptic feedback in da Vinci robot. One of them is getting accurate force readings from the patient side manipulator (PSM). To address this issue, we are trying to create force-feedback device, that can be easily added to the existing surgery system.