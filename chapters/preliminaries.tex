%% FRONTMATTER

\begin{frontmatter}

	% generate title
	\maketitle
	
	\pagebreak
	\vskip 2em
			 {Approved by:\par}
			 \vskip 1em
			 \vspace{10 mm}
			 \ssp\rule[-2pt]{8cm}{0.5pt}\par
			 {Prof. Gregory S. Fischer, Advisor\par}
			 {Worcester Polytechnic Institute\par}
	
			 \vskip 1em
			 \vspace{10 mm}
			 \ssp\rule[-2pt]{8cm}{0.5pt}\par
			 {Prof. Karen Troy, Committee Member\par}
			 {Worcester Polytechnic Institute\par}
	
			 \vskip 1em
			 \vspace{10 mm}
			 \ssp\rule[-2pt]{8cm}{0.5pt}\par
			 {Prof. Loris Fichera, Committee Member\par}
			 {Worcester Polytechnic Institute\par}
	
	\begin{abstract}
Teleoperated robotic surgical systems such as daVinci are widely used for laparoscopic surgeries. The currently available daVinci system does not provide haptic feedback. Prior research has shown that the addition of haptic feedback improves surgeons' performance during minimally invasive surgeries. Other authors have implemented haptic feedback in the daVinci robot by placing sensors on the surgical tools, using visual force estimation, and measuring proximal guide wire forces. However, they have faced issues with biocompatibility, time delay, low accuracy, and repeatability. In this work, two strain gauge force-sensing devices were created for the patient side manipulator of the daVinci surgical robot. These devices were designed to be easily added to the existing system. The device mounted on the cannula measures the X-Y components of the forces applied to the tool, and the device mounted on the sterile adapter measures the Z-component of the force. These devices are used for the real-time force feedback in the daVinci robot. The proposed system has high sensitivity and resolution, matches the required force measurement range, and has high signal-to-noise ratio, which implies high signal quality. However, the absolute errors of the currently built devices are high. % ($error_x = -0.075 \pm 0.427 N$, $error_y = 0.1 \pm  0.9 N$, $error_z = -0.005 \pm 0.952 N$)
This work demonstrates fast 3-DOF force measurements on the daVinci robot without any robot modifications. While the present system has significant systematic errors, these can be mitigated by altering the mechanical design to reduce hysteresis and improve the accuracy of the system.

	\end{abstract}
	
	\begin{acknowledgment}
	
	I would like to express my gratitude to everybody in the world.
	
	\end{acknowledgment}
	
	\begin{dedication}
	
	This dissertation is dedicated to everybody in the world.
	
	\end{dedication}
	
	% generate table of contents
	\tableofcontents
	
	% generate list of tables
	\listoftables
	
	% generate list of figures
	\listoffigures
	
	Disclaimer: certain materials are included under the fair use exemption of the U.S. Copyright Law and have been prepared according to the fair use guidelines and are restricted from further use.
	
	\pagebreak
	\section*{Acronyms}
	\begin{acronym}
	\acro{PSM}{Patient Side Manipulator}
	\acro{DOF}{Degrees of Freedom}
	\acro{PCB}{Printed Circuit Board}
	\acro{ROS}{Robot Operating System}
	\acro{SD}{Standard Deviation}
	\acro{SNR}{Signal-to-noise Ratio}
	\acro {GF}{Gauge Factor}
	
	\end{acronym}
	
	\end{frontmatter}
	
	