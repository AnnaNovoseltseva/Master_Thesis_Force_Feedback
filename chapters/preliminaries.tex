%% FRONTMATTER

\begin{frontmatter}

	% generate title
	\maketitle
	
	\pagebreak
	\vskip 2em
			 {Approved by:\par}
			 \vskip 1em
			 \vspace{10 mm}
			 \ssp\rule[-2pt]{8cm}{0.5pt}\par
			 {Prof. Gregory S. Fischer, Advisor\par}
			 {Worcester Polytechnic Institute\par}
	
			 \vskip 1em
			 \vspace{10 mm}
			 \ssp\rule[-2pt]{8cm}{0.5pt}\par
			 {Prof. Karen Troy, Committee Member\par}
			 {Worcester Polytechnic Institute\par}
	
			 \vskip 1em
			 \vspace{10 mm}
			 \ssp\rule[-2pt]{8cm}{0.5pt}\par
			 {Prof. Loris Fichera, Committee Member\par}
			 {Worcester Polytechnic Institute\par}
	
	\begin{abstract}
	
	statement of problem
	In order to get force feedback for da Vinci surgical robot was 
	
	procedure
	designed 2-DOF force-sensing instrument for patent side manipulator. In a future it can be used for haptic feedback.
	
	results
	Designed instrument showed that accuracy and sensitivity, which complies to mostly all requirements.
	
	conclusions
	Many things to be changed (add Z-direction, biocompatibility)
	
	
	
	
	\end{abstract}
	
	\begin{acknowledgment}
	
	I would like to express my gratitude to everybody in the world.
	
	\end{acknowledgment}
	
	\begin{dedication}
	
	This dissertation is dedicated to everybody in the world.
	
	\end{dedication}
	
	% generate table of contents
	\tableofcontents
	
	% generate list of tables
	\listoftables
	
	% generate list of figures
	\listoffigures
	
	Disclaimer: certain materials are included under the fair use exemption of the U.S. Copyright Law and have been prepared according to the fair use guidelines and are restricted from further use.
	
	\pagebreak
	\section*{Acronyms}
	\begin{acronym}
	\acro{PSM}{Patient Side Manipulator}
	\acro{DOF}{Degrees of Freedom}
	\acro{PCB}{Printed Circuit Board}
	\acro{ROS}{Robot Operating System}
	\acro{SD}{Standard Deviation}
	
	\end{acronym}
	
	\end{frontmatter}
	
	