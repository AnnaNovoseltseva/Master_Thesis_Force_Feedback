%% FRONTMATTER

\begin{frontmatter}

	% generate title
	\maketitle
	
	\pagebreak
	\vskip 2em
			 {Approved by:\par}
			 \vskip 1em
			 \vspace{10 mm}
			 \ssp\rule[-2pt]{8cm}{0.5pt}\par
			 {Prof. Gregory S. Fischer, Advisor\par}
			 {Worcester Polytechnic Institute\par}
	
			 \vskip 1em
			 \vspace{10 mm}
			 \ssp\rule[-2pt]{8cm}{0.5pt}\par
			 {Prof. Karen Troy, Committee Member\par}
			 {Worcester Polytechnic Institute\par}
	
			 \vskip 1em
			 \vspace{10 mm}
			 \ssp\rule[-2pt]{8cm}{0.5pt}\par
			 {Prof. Loris Fichera, Committee Member\par}
			 {Worcester Polytechnic Institute\par}
	
	\begin{abstract}
Teleoperated robotic surgical systems such as daVinci are widely used for laparoscopic surgeries. The currently available daVinci system does not provide haptic feedback. Prior research has shown that the addition of haptic feedback improves surgeons' performance during minimally invasive surgeries. Other authors have implemented haptic feedback in the daVinci robot by placing sensors on the surgical tools, using visual force estimation, and measuring proximal guide wire forces. However, issues with biocompatibility, time delay, low accuracy, and repeatability make them impractical for clinical use. In this work, two strain gauge force-sensing devices were created for the patient side manipulator of the daVinci surgical robot. These devices were designed to be easily added to the existing system. The device mounted on the cannula measures the X-Y components of the forces applied to the tool, and the device mounted on the sterile adapter measures the Z-component of the force. These devices are used for the real-time force feedback in the daVinci robot. The proposed system has high sensitivity and resolution, matches the required force measurement range, and has high signal-to-noise ratio, which implies high signal quality. However, the absolute errors of the currently built devices are high due to the manufacturing techniques used on the prototype that could be improved upon for a deployed device.
This work demonstrates fast 3-DOF force measurements on the daVinci robot without any robot or instrument modifications. While the present system has significant systematic errors, these can be mitigated by altering the mechanical design to reduce hysteresis and improve the accuracy of the system.

	\end{abstract}
	
	\begin{acknowledgment}
I would first like to thank my thesis advisor Professor Gregory Fischer. 
Prof. Fischer always guided me throughout my research by giving me valuable and timely advisement.

I would also like to thank the experts who were involved in the validation of this research project: Prof. Karen Troy and Prof. Loris Fichera. Without their participation and input, the validation could not have been successfully conducted.

I would also like to thank my lab mates Adnan Munawar, Radian Azhar Gondokaryono, Paulo Carvalho, Joseph Schornak, Abhishek Kashyap, Chris Nycz for giving me with valuable pieces of advice.

Finally, I must express my profound gratitude to my parents and to my boyfriend for providing me with unfailing support and continuous encouragement throughout my years of study and through the process of researching and writing this thesis. This accomplishment would not have been possible without them. Thank you.
	\end{acknowledgment}
	
	% generate table of contents
	\tableofcontents
	
	% generate list of tables
	\listoftables
	
	% generate list of figures
	\listoffigures
	
	Disclaimer: certain materials are included under the fair use exemption of the U.S. Copyright Law and have been prepared according to the fair use guidelines and are restricted from further use.
	
	\pagebreak
	\section*{Acronyms}
	\begin{acronym}
	\acro{PSM}{Patient Side Manipulator}
	\acro{DOF}{Degrees of Freedom}
	\acro{CAD/CAM}{Computer-Aided Design/Computer-Aided Manufacturing}
	\acro{QTC Pills} {Quantum Tunneling Composite Pills}
	\acro{RMS}{Root Mean Square}
	\acro{PCB}{Printed Circuit Board}
	\acro{ROS}{Robot Operating System}
	\acro{SD}{Standard Deviation}
	\acro{SNR}{Signal-to-noise Ratio}
	\acro{GF}{Gauge Factor}
	\acro{ADC}{Analog to Digital Converter}
	\acro{FFT}{Fast Fourier transform}
	\acro{RMSE}{Root Mean Square Error}
	\end{acronym}
	
	\end{frontmatter}
	
	